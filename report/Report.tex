\documentclass[12pt]{article}
\usepackage{bm}

\begin{titlepage}

\title{\textbf{Optimum Loft Angle for Greatest Carry Distance}}
\author{\textbf{Group D}\\
		Alison McIntosh\\
		Emily Dark\\
		Henry Archer\\
		Kyle Stewart\\
		Stuart Ballantyne}
\date{}

\end{titlepage}

\begin{document}

\maketitle

\section{Abstract}
...


\section{Response}
...

\section{Theory and model}

\subsection{Assumptions}
The following assumptions are made throughout the report and model:
\begin{description}
  \item[$\cdot$] golf course is level and has no effect on trajectory;
  \item[$\cdot$] height of the tee is negligible;
  \item[$\cdot$] gravitational field strength is constant (9.81 $N\cdot kg^{-1}$) and does not flucuate with height;
  \item[$\cdot$] driver is roughly a flat plate and strikes the ball precisely at the center, with no draw or fade.
\end{description}

\subsection{Impact}
...

\subsection{Flight}
A simple golf ball experiencing only weight may be modelled with the following system of differential equations:
\begin{equation}
\frac{\partial v_x}{\partial t}=0
\end{equation}
\begin{equation}
\frac{\partial v_y}{\partial t}=-g
\end{equation}
where $g$ is the gravitional field strength.



\end{document}
\grid
